\documentclass[a4paper,12pt,twoside]{book}

%\usepackage{fontspec} Packages
\usepackage[utf8]{inputenc}
\usepackage{titlesec}
\usepackage{emptypage} % Per tal que no surtin els números en les pàgines en blanc.
\usepackage{blindtext}
\usepackage[spanish]{babel} % Idioma del document en español.
\usepackage[tmargin=3cm, lmargin=3cm, rmargin=3cm, bmargin=3cm, marginparsep=0.3cm, marginparwidth=2.4cm]{geometry}
\usepackage[sorting=none]{biblatex} % Per la bibliografia a més que els números surtin per l'ordre en què apareixen, no en l'ordre que s'ha citat al llarg del document.

%\setmainfont{Arial}

\pagenumbering{gobble}

%DEFINICIONS per tal de no haver d'escriure-ho cada vegada.

\def\upf{Universitat Pompeu Fabra }

%BIBLIOGRAFIA

\addbibresource{bibliografia.bib}

%Titols

\title{\textbf{Plan de Negocio de GesTeam}}

\author{Aurel Ioan Patrutiu \\ Jordi Bosch Garcia}

\date{\today \\ \upf}

% COMENÇA EL REPORT

\begin{document}

\maketitle % Crea la pàgina del títol amb les etiquetes de sobra, \title, \author i \date.

\cleardoublepage % Neteja de pàgines, en ser un llibre ja la deixa en blanc en cas de necessitat.

\pagenumbering{Roman} % Nombre de pàgina en números Romans.

\chapter*{Dedicatória}
\addcontentsline{toc}{chapter}{Dedicatória}
\input{Chapters/Dedicatoria}

\chapter*{Agradecimientos}
\addcontentsline{toc}{chapter}{Agradecimientos}
\input{Chapters/Agradecimientos}

\chapter*{Resumen}
\addcontentsline{toc}{chapter}{Resumen}
\input{Chapters/Resumen}

\chapter*{Prefacio}
\addcontentsline{toc}{chapter}{Prefacio}
\input{Chapters/Prefacio}

\tableofcontents % Taula de continguts

\cleardoublepage

\pagenumbering{arabic} % Nombre de la pàgina en números Àrabs.

\chapter{Introducción} % Ja es posen en la Taula de continguts.
\input{Chapters/Introduccion}

\printbibliography
\addcontentsline{toc}{chapter}{Bibliografía}

\end{document}
