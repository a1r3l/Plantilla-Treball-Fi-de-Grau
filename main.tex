\documentclass[a4paper,12pt,twoside]{book}

%\usepackage{fontspec} Packages
\usepackage[catalan]{babel} % Idioma del document en català.
\usepackage[utf8]{inputenc}
\usepackage{emptypage} % Per tal que no surtin els números en les pàgines en blanc.
\usepackage{blindtext}
\usepackage[tmargin=3cm, lmargin=3cm, rmargin=3cm, bmargin=3cm, marginparsep=0.3cm, marginparwidth=2.4cm]{geometry}
\usepackage[sorting=none]{biblatex} % Per la bibliografia a més que els números surtin per l'ordre en què apareixen, no en l'ordre que s'ha citat al llarg del document.
\usepackage{svg} %Imporat imatges amb l'extenció svg
\usepackage{pdfpages} %Importar la portada en pdf
\usepackage{hyperref} %Les referències quan cliques funcionen
\usepackage{import}
\usepackage{graphicx}
\usepackage{subfiles} %Per importar la pàgina del títol
\usepackage{fancyhdr} %Que quedi maco
\usepackage{nameref} %Referències amb el nom que s'ha posat
\usepackage[chapter]{algorithm} %Que surti el chapert davant del nombre, igual que el lot o lof
\usepackage[noend]{algpseudocode}


\pagestyle{fancy}
\fancyhf{}
\fancyhead[LE,RO]{\leftmark}
\fancyhead[RE,LO]{\thepage}

\pagenumbering{gobble}

%COMANDS comandes personalitzades

\newcommand{\capitol}[1]
{
  \addcontentsline{toc}{chapter}{#1} % Per tal d'afegir a la Table of contents, amb el nom Agraïments.
}

\newcommand{\figura}[4]
{
  \begin{figure}[h]
      \begin{center}
          \includegraphics[scale=.#3]{#1}
          \caption{#2}
          \label{fig:#4}
      \end{center}
  \end{figure}
}


%DEFINICIONS per tal de no haver d'escriure-ho cada vegada.

\usepackage{acro}

\DeclareAcronym{upf}{
  short = UPF,
  long = Universitat Pompeu Fabra,
  class = abbrev
}

%%% Local Variables:
%%% mode: latex
%%% TeX-master: "../report"
%%% End:


%BIBLIOGRAFIA

\addbibresource{bibliografia.bib}


\begin{document}

\includepdf[pages=-]{portada}

%TITOL
\subfile{Titol}

\cleardoublepage % Neteja de pàgines, en ser un llibre ja la deixa en blanc en cas de necessitat.

\pagenumbering{Roman} % Nombre de pàgina en números Romans.

\chapter*{Agraïments} %* Per tal que no surti en el Table of contents, així la primera secció començarà amb la Introducció.
\capitol{Agraïments} %Comanda nova, mirar a l'arxiu de Comands
\input{Chapters/Agraiments} % Carpeta i arxiu el qual es vol afegir, així el main només queda amb l'esquelet i el contingut va repartir en els diferents arxius.

\tableofcontents % Taula de continguts

\listoffigures
\capitol{Índex de figures}

\listoftables
\capitol{Índex de taules}

\listofalgorithms
\capitol{List of algorithms}

\printacronyms[include-classes=abbrev,name=Notations,heading=chapter*]
\capitol{Abbreviations}

\pagenumbering{arabic} % Nombre de la pàgina en números Àrabs.

\chapter{Introducció} % Ja es posen en la Taula de continguts.
\section{Seccions principals de l'article, vindria a ser el Title 2 del Google Docs.}
El Title 1 en aquest cas correspondri a l'article,  en el main principal.
\subsection{Subseccions com el Títol 3 del Google Docs.}

\newpage %Veure el fancy com queda

Totes aquestes seccions quedaran posades en la Taula de continguts \cite{autor} % Per tal de poder citar una des de la bibliografia.

\figura{upf_word_imp.jpg}{\acl{upf}}{5}{1}


\chapter{Algorismes}
\begin{algorithm}[H]
  \caption{Exemple}
  \label{alg:Exemple}
  \begin{algorithmic}[1]
  	\Procedure{Exemple}{Doncs podrien estar buits els paràmetres}
            \For{i $\to \infty$}
                \State El TFG no s'acaba
            \EndFor
            \State \Return{Això no acaba TT}
        \EndProcedure
   \end{algorithmic}
\end{algorithm}


\printbibliography
\capitol{Bibliografia}

\end{document}
